% !TeX program = xelatex
\PassOptionsToPackage{quiet}{fontspec}
\documentclass[12pt,a4paper,UTF8]{ctexart}
\CTEXsetup[format={\Large\bfseries}]{section} %% section 左对齐
\usepackage{zjureport}

% ++
\usepackage{float} %解决figure中使用[H]绝对定位的问题
\usepackage{tikz}
\usetikzlibrary{positioning, shapes.geometric, calc}
\usepackage{listings}
\usepackage{xcolor} % 引入颜色宏包
\usepackage{tabularx} % 制表
\usepackage{booktabs}
\usepackage{graphicx}
\usepackage{subfigure}

\lstset{
    language=Python,  % 设置语言为Python
    basicstyle=\ttfamily,  % 设置基本字体样式
    tabsize=4,  % 设置制表符宽度
    numbers=left,  % 行号显示在左边
    numberstyle=\tiny\color{gray},  % 行号样式
    breaklines=true,  % 自动换行
    frame=single,  % 代码块框架
    extendedchars=false, % 禁止断页,未解决!
}

\usepackage{url}

% algorithm类不用引入,直接使用cls的al2e并修改即可
\usepackage{algorithm2e}
\newlength{\commentWidth}
\setlength{\commentWidth}{6cm}  %距离右边的宽度,越大则注释越靠左
\newcommand{\atcp}[1]{\tcp*[r]{\makebox[\commentWidth]{#1\hfill}}}
\RestyleAlgo{ruled}
\SetKwComment{Comment}{/* }{ */}
\SetKwInput{KwData}{数据}
\SetKwInput{KwResult}{结果}
\SetKwInOut{KwIn}{输入}
\SetKwInOut{KwOut}{输出}

%--------------------信息设置---------------------------%
\mytitle{论摸鱼中的工程创新前沿}
\myauthor{摸个鱼先}
\mystudentid{31415926}
\mycourse{《摸鱼学导论》课程论文}
\myschool{摸鱼大学}
\mygrade{-1}
\myproject{摸鱼项目}
\mymajor{摸鱼科学与工程}
\myteacher{xxx \ 教授}


\begin{document}
\cover
\thispagestyle{empty} % 首页不显示页码

%%-----目录页----------%%
% \newpage
% \tableofcontents
\newpage
\setcounter{page}{1}%从下面开始编页码

%--------------------正文---------------------------%

\begin{center}
   % \title{ \Huge \textbf{{\themytitle}}}
    \title{\Huge \textcolor[rgb]{0, 0.44, 0.75}{\textbf{\themytitle}}}
\end{center}

\begin{abstract}
your abstract here.
\end{abstract}


% Chapter 1

\section{绪论}

这里概括性用一段话介绍一下本研究的重要性与意义。

\subsection{研究背景与意义}
NeRF是自2019年兴起的基于神经隐式表示的新型视觉合成方法\cite{mildenhall_nerf_2020}。

\subsection{研究内容}

如下流程图~\ref{fig:paper-structure}所示:

\begin{figure}[H]
    \centering
    \begin{tikzpicture}[node distance=10pt]
      \node[draw, rounded corners]                        (start)   {国内外研究现状研究};
      \node[draw, below=of start]                         (step 1)  {balabala};
      \node[draw, below=of step 1]                        (step 2)  {balabala};
      \node[draw, below=of step 2]                        (step 3)  {balabala};
      \node[draw, diamond, aspect=2, below=of step 3]     
      (choice)  {balabala};
      \node[draw, right=30pt of choice]                   (step x)  {balabala};
      \node[draw, rounded corners, below=20pt of choice]  (end)     {balabala};
      
      \draw[->] (start)  -- (step 1);
      \draw[->] (step 1) -- (step 2);
      \draw[->] (step 2) -- (step 3);
      \draw[->] (step 3) -- (choice);
      \draw[->] (choice) -- node[left]  {Yes} (end);
      \draw[->] (choice) -- node[above] {No}  (step x);
      \draw[->] (step x) -- (step x|-step 3) -> (step 3);
    \end{tikzpicture}
    \caption{本文研究工作流程图}\label{fig:paper-structure}
\end{figure}


\section{公式与图片}
请用xelatex编译,推荐使用overleaf。本模板参考了:\href{https://www.philfan.cn/Tools/latex/#overleaf}{PhilFan's Notebook-Latex备忘录}的相关模板,在此致以感谢!

\subsection{公式}
具体的,光线的关系如下式\eqref{eq:eq-1}:
\begin{equation}
\mathbf{C}(\mathbf{r})=\int_{t_n}^{t_f}T(t)\sigma(\mathbf{r}(t))\mathbf{c}(\mathbf{r}(t),\mathbf{d})dt,\mathrm{~where~}T(t)=\exp\left(-\int_{t_m}^t\sigma(\mathbf{r}(s))ds\right).
\label{eq:eq-1}
\end{equation}

\subsection{图片}
而光线的关系图则如下图\eqref{fig:zju_char}所示:
\begin{figure}[hb]
  \centering
  \includegraphics[width=0.4\textwidth]{figures/zju_char.png}
  \caption{zju\_char\cite{mildenhall_nerf_2020}}
  \label{fig:zju_char}
\end{figure}

\newpage
\reference
\end{document}